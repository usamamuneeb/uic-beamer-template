\documentclass{beamer}
\usetheme{uic}
\usepackage{amsfonts,amsmath,oldgerm,algorithmic,algorithm}
\usepackage[font=small,labelfont=bf]{caption} % Required for specifying captions to tables and figures

\newcommand{\hrefcol}[2]{\textcolor{uihteal}{\href{#1}{#2}}}
\newcommand{\testcolor}[1]{\colorbox{#1}{\textcolor{#1}{test}}~\texttt{#1}}

% Please see Section 18.1 of Beamer User Guide for all the options \usefonttheme provides
\usefonttheme[onlymath]{serif}
% \usefonttheme{serif} % use this if you would like Serif font throughout (and not just for math)

\title{Get started with the UIC Beamer Theme}


\titlebackground*{images/uic_lockup.png}

% NOTE 1: The asterisk splits the background image. This option is good
% for logo-based backgrounds. If you use an image based background
% it's recommended to not split it:

% \titlebackground{images/uic_seo.jpg}

% NOTE 2: If you use a title background that does not have a logo, you might
% want to enable logo in the top left.
% To do that, simply comment out this line
\themecolor{lightnologo}

\subtitle{Using \LaTeX\ to prepare slides}
% This can be adjusted accordingly for longer titles
\newcommand{\titleboxwidth}{0.45\textwidth}

\author{\href{mailto:umunee2@uic.edu}{Usama Muneeb}}
\date{\today}

\begin{document}
\maketitle
\themecolor{light} % reverts to a logo based theme (if you disabled it for title page)

% enabled after title page creation (i.e. after \maketitle)
\footlinecolor{uicblue}


\begin{frame}[fragile]{Beamer for UIC presentations}
\framesubtitle{A quick start}
If you would like \LaTeX\ in your presentation, Beamer is a great way to go!
\begin{itemize}
\item Beamer has a detailed
\hrefcol{https://www.ctan.org/tex-archive/macros/latex/contrib/beamer/doc/beameruserguide.pdf}{user
 manual}, but we will go over the most common features.
\item This template is designed for a 16:9 aspect ratio, which is the default in PowerPoint and the most common amongst projectors.
\end{itemize}
\begin{itemize}[<+->]
\item The most common of all slide types involve bulleted points, like these.
\begin{itemize}
\item Changing \verb|\begin{itemize}| to \verb|\begin{itemize}[<+->]| will make the bullet points unroll in a sequence.
\end{itemize}
\end{itemize}
\end{frame}


\begin{chapter}[images/uic_seo.jpg]{uicblue}{Slide Layouts}
\textit{Some standard slide layouts}
\end{chapter}



\begin{frame}[fragile]{Images, columns and tips for good slides}
\framesubtitle{\textbf{Do use} this optional subtitle where appropriate}
\begin{columns}
\begin{column}{0.7\textwidth}
\begin{itemize}
    \item \textbf{Do keep} a good balance between text and figures.
    \begin{itemize}
        \item \textbf{Do use} nested lists wherever appropriate.
    \end{itemize}
    \item \textbf{Do use} the \verb|\underbrace| and \verb|\overbrace| commands to help explain your equations:
\begin{equation*}
\begin{aligned}
\max_{\lambda, \nu}. \quad & \underbrace{\inf_x \overbrace{L(x, \lambda, \nu)}^\text{Lagrangian}}_\text{Lagrange Dual Function, $g(\lambda, \nu)$}\\
\textrm{s.t.} \quad & \lambda \geq 0
\end{aligned}
% \max_{\lambda, \nu}. \underbrace{\inf_x \overbrace{L(x, \lambda, \nu)}^\text{Lagrangian}}_\text{Lagrange Dual Function, $g(\lambda, \nu)$}, \quad \textrm{s.t.} \quad \lambda \geq 0
\end{equation*}
\end{itemize}


\end{column}
\begin{column}{0.3\textwidth}
\includegraphics[width=\textwidth]{plots/min_fuel.png}
\captionof{figure}{Minimum fuel trajectory}
\end{column}
\end{columns}
\end{frame}


\begin{frame}[fragile]{More with bullet points}
\framesubtitle{Alerts and repeats}
Sections 12.1 through 12.3 of the \hrefcol{https://www.ctan.org/tex-archive/macros/latex/contrib/beamer/doc/beameruserguide.pdf}{Beamer user
manual} demonstrate many more features, like alerts and repeats.
\begin{itemize}
\item The \verb|\alert{}| feature can be particularly useful.
\item Like \alert<2>{this}.
\item The highlighting may span multiple slides in a frame.
    \begin{itemize}
    \item Previous one did not, but \alert<3->{this one} will.
    \end{itemize}
\item This is accomplished by the \alert<4>{hyphen} (i.e. \texttt{-}) in the \verb|\alert<3->| command.
\end{itemize}
\end{frame}

\begin{frame}{Another images/columns example}
\begin{columns}
\begin{column}{0.35\textwidth}
\includegraphics[width=\textwidth]{plots/min_fuel.png}
\captionof{figure}{Minimum fuel trajectory}
\end{column}
\begin{column}{0.3\textwidth}
\textbf{Left:} Solving for optimal fuel consumption \textbf{Right:} Solving for optimal time taken
\end{column}
\begin{column}{0.35\textwidth}
\includegraphics[width=\textwidth]{plots/min_time.png}
\captionof{figure}{Minimum time trajectory}
\end{column}
\end{columns}
\end{frame}


\begin{frame}{Blocks}
Some content will just look better in blocks.
\begin{columns}
\begin{column}{0.4\textwidth}
\begin{block}{Block title}
This is a \texttt{block}. Its color will match the color of the footline. The \texttt{block} environment is native to Beamer.
\end{block}
\end{column}
\begin{column}{0.6\textwidth}
\begin{colorblock}[black]{chicagoblue}{Block title}
This Beamer theme also provides a \texttt{colorblock} environment which is a wrapper on top of the usual \texttt{block} environment in Beamer. It allows you to specify a custom background and font color.
\end{colorblock}
\end{column}
\end{columns}

\vspace{3ex}
Here's an example usage:

\begin{columns}
\begin{column}{0.5\textwidth}
\begin{block}{The discrete case}
...
\end{block}
\end{column}
\begin{column}{0.5\textwidth}
\begin{colorblock}[black]{chicagoblue}{The continuous case}
...
\end{colorblock}
\end{column}
\end{columns}
\end{frame}


\begin{frame}{Blocks for theorems}
Beamer also uses blocks by default to wrap theorems.
\begin{theorem} This is a theorem. \end{theorem}
\end{frame}


\begin{sidepic}{images/uic_quad.pdf}{Side-Picture Slides}
This Beamer theme also provides a \texttt{sidepic} environment which is a wrapper on top of the \texttt{frame} environment.
\begin{itemize}
\item It has an optional \texttt{image} argument which will help you achieve a layout of this type.
\end{itemize}
\end{sidepic}


\renewcommand{\algorithmicrequire}{\textbf{Input:}}
\renewcommand{\algorithmicensure}{\textbf{Output:}}
\newcommand*\CALL[2]{\textsc{#1}(#2)}
% \newcommand*\ANNOTATE[1]{\hfill\{#1\}}%
% \newcommand*\ANNOTATE[1]{\hskip3em$\rightarrow$ #1}
\newcommand*\ANNOTATE[1]{\hfill\(\triangleright\) #1}%

\footlinecolor{}

\begin{frame}{Pseudocode Example}
\begin{algorithm}[H]
\fontsize{8}{1}\selectfont
\caption{Bellman-Kalaba (adapted from \hrefcol{https://www.ctan.org/tex-archive/macros/latex/contrib/algorithmicx/algorithmicx.pdf}{algorithmicx documentation}).}
\begin{algorithmic}
\REQUIRE $G$, $u$, $l$, $p$
% \ENSURE optional output
\FORALL {$v \in V(G)$}
    \STATE $l(v) \leftarrow \infty$
\ENDFOR
\STATE $l(u) \leftarrow 0$
\REPEAT
    \FOR {$i \leftarrow 1, n$}
        \STATE $min \leftarrow l(v_i)$
        \FOR {$j \leftarrow 1, n$}
            \IF {$min > \CALL{Edge}{v_i, v_j} + l(v_j)$}
                \STATE $min \leftarrow \CALL{Edge}{v_i, v_j} + l(v_j)$ \ANNOTATE{Example comment}
                \STATE $p(i) \leftarrow v_j$
            \ENDIF
        \ENDFOR
        \STATE $l’(i) \leftarrow min$
    \ENDFOR
    \STATE $changed \leftarrow l \not= l’$
    \STATE $l \leftarrow l’$
\UNTIL{$\neg changed$}
% \RETURN optional return
\end{algorithmic}
\end{algorithm}
\end{frame}


\footlinecolor{uicblue}

% \renewcommand{\titlemanualoffset}{-0.4ex}
\begin{frame}[fragile]{Common issues (this is $\hat P$ laceholder math)}
\begin{itemize}
\item Math in title will cause its height to be inconsistent with that of other frames (try viewing these example slides in a full screen mode)
\begin{itemize}
\item This theme provides a \verb|\titlemanualoffset| command which can be used to manually adjust this offset.
\begin{itemize}
\item Putting \verb|\renewcommand{\titlemanualoffset}{<some calibrated offset>}| before \verb|\begin{frame}| for this frame will fix this issue.
\item \textbf{Make sure to zero it out after} \verb|\end{frame}|
\end{itemize}
\end{itemize}
\item If you use \verb|\verb| command (or \verb|\begin{verbatim}| environment) in a frame, make sure to specify the \texttt{[fragile]} option in \verb|\begin{frame}| otherwise the compiler will run into problems.
\begin{itemize}
\item In most cases, \verb|\texttt| can be used instead of \verb|\verb|, and it does not require the \texttt{[fragile]} option.
\end{itemize}
\end{itemize}
\end{frame}
\renewcommand{\titlemanualoffset}{0.0ex}



\themecolor{lightnologo} % since we are using a logo based background, disable the (duplicate) logo in top left
\begin{chapter}[images/uic_lockup_white.png]{uicblue}{Personalization}
\textit{Fonts and color schemes}
\end{chapter}
\themecolor{light} % restore the logo in top left


\begin{frame}[fragile]{If you need more space in slides}
If you would like more space, you can control the font size by specifying an option in the \verb|\documentclass| command at the beginning of this file.
\begin{itemize}
    \item Use \verb|\documentclass[smaller]{beamer}| to reduce font size.
    \item For all options, refer to Section 18.2.1 of the \hrefcol{https://www.ctan.org/tex-archive/macros/latex/contrib/beamer/doc/beameruserguide.pdf}{Beamer user manual}.
\end{itemize}
\end{frame}


\begin{frame}[fragile]{Color schemes}
Three fundamental features greatly affect the look and feel of your slides. Two of them can be controlled with the \verb|\themecolor| command (provided by this theme).
\begin{itemize}
\item The background color.
\begin{itemize}
    \item The default is \verb|light| theme, which has a light background, which we are currently using.
\end{itemize}
\item The foreground color.
\begin{itemize}
    \item The \verb|light| theme has a \verb|dark| foreground (i.e. text color) and vice versa.
\end{itemize}
\item The third, the footer color can be set using \verb|\footlinecolor| command.
\begin{itemize}
    \item We are currently using \testcolor{uicblue} color for our footer.
    \item The default is no footline, but I believe page numbers are incredibly helpful for your audience to ask questions later.
\end{itemize}
\end{itemize}
\end{frame}


\themecolor{dark}
\footlinecolor{uicred}
\begin{frame}[fragile]{Color schemes}
\begin{itemize}
\item I just called \verb|\themecolor{dark}| before \verb|\begin{frame}| for this slide.
\begin{itemize}
    \item Ideally you should set the theme globally in the preamble (i.e. before \verb|\begin{document}|). Default is light.
\end{itemize}
\item You can also change the footer color with \verb|\footlinecolor{color}|, as we did for this slide.
\item All the changes we talked about here happen for the current \textbf{as well as subsequent} frames.
\begin{itemize}
    \item I will manually revert all the changes for the next slide.
\end{itemize}
\end{itemize}
\end{frame}


\themecolor{light}
\footlinecolor{uicblue}

\begin{frame}[fragile]{UIC Colors}
\begin{itemize}
\item Font color can be set with the \verb|\textcolor{<color name>}{text}| command.
\item The colors are defined in the \texttt{uiccolor} package, in accordance with the \hrefcol{https://marketing.uic.edu/marketing-toolbox/university-style-guide/visual-elements/}{UIC Visual Elements guide}:
\begin{itemize}
\item UIC's primary colors are Navy Pier Blue (\testcolor{uicblue}) and Fire Engine Red (\testcolor{uicred}).
\item UIC's secondary colors are Chicago Blue (\testcolor{chicagoblue}), UI Health Teal (\testcolor{uihteal}) and Champions Gold (\testcolor{championsgold}).
\item UIC's neutral colors are Expo White (\testcolor{expowhite}) and Steel Gray (\testcolor{steelgray} or \testcolor{steelgrey}).
\end{itemize}
\item A \emph{simpler alternative} to font colors is often just \verb|\emph{}|.
\end{itemize}
\end{frame}


\begin{frame}[fragile]{Fonts}
Fonts are categorized as \textrm{Serif} and \textsf{Sans-Serif} (see \hrefcol{https://www.adobe.com/creativecloud/design/discover/serif-vs-sans-serif.html}{this link} for when to use which).

\begin{itemize}
    \item \textbf{\textsf{Open Sans}} and \textbf{\textrm{IBM Plex Serif}} have been provided with this template.
    \item Beamer uses \textsf{Sans-Serif} mode by default.
    \begin{itemize}
        \item To switch to Serif mode, please change \verb|\usefonttheme[onlymath]{serif}| to \verb|\usefonttheme{serif}| in the preamble.
    \end{itemize}
    \item If you want to mix and match Serif and Sans-Serif
    \begin{itemize}
        \item You can still typeset in Serif font in Sans-Serif mode (and vice versa) using \verb|\textrm| and \verb|\textsf| commands.
    \end{itemize}
\end{itemize}
For more options with fonts, you need to modify the self explanatory \verb|uicfont.sty| file. Next few slides will be helpful in this regard.
\end{frame}


\begin{frame}{Font formats over time}
Here are the \textbf{old} formats (still used in the LaTeX ecosystem):

\begin{itemize}
\item \emph{PostScript Type 1} format developed by Adobe in the 1980s
\begin{itemize}
    \item Authoring support \hrefcol{https://helpx.adobe.com/fonts/kb/postscript-type-1-fonts-end-of-support.html}{officially ended by Adobe in January 2023}
\end{itemize}
\item \emph{TrueType format} (\texttt{.ttf}) developed by Apple, also in 1980s, licensed to Microsoft
\end{itemize}

The \textbf{new} font formats are \emph{OpenType}, developed jointly by Microsoft and Adobe in the 1990s as an extension of Apple's TrueType font format. \emph{OpenType} fonts are either

\begin{itemize}
\item \emph{PostScript flavor OpenType} (\texttt{.otf}) that supercedes Adobe's PostScript Type 1
\item \emph{TrueType flavor OpenType} (also \texttt{.ttf}!) that supercedes Apple's TrueType
    \begin{itemize}
    \item Distinguishing them from the old TrueType format is non trivial, since they have the same file extension.
    \end{itemize}
\end{itemize}
\end{frame}


\begin{frame}{Font formats compatibility}
The new \emph{OpenType} fonts have several new features over the old formats. They are however, supported natively only by XeLaTeX and LuaLaTeX compilers. The most popular pdfLaTeX compiler has limited support for them:
\begin{itemize}
\item \emph{OpenType} fonts are not pdfLaTeX-ready by default.
\item They require the production of TeX font metrics and other ancillary files in order to be used with LaTeX.
\item The production of these files is not easy for the average user.
\end{itemize}
Consequently, if using pdfLaTeX, the most convenient option is to restrict yourself to the pdfLaTeX-ready font packages in your TeX distribution (exhaustive list provided in the \hrefcol{https://tug.org/FontCatalogue/}{LaTeX Font Catalog}).
\end{frame}


\begin{frame}{Font formats compatibility}
If you want to use fonts beyond those in the \hrefcol{https://tug.org/FontCatalogue/}{LaTeX Font Catalog} (for example, fonts from Google Fonts), the recommended approach is to use XeLaTeX.
\begin{itemize}
\item XeLaTeX is set as the default compiler on the Overleaf version of this template.
\item The fonts included in the \texttt{fonts} folder are \emph{TrueType flavor OpenType} fonts.
\item If pdfLaTeX is used, font packages from your TeX distribution will be used instead.
\end{itemize}
UIC signature font is \textbf{Theinhardt}, which is a proprietary font.
\begin{itemize}
\item UIC \hrefcol{https://marketing.uic.edu/marketing-toolbox/uic-fonts/}{provides Theinhardt}, but proprietary fonts sometimes do not permit PDF embedding which might cause substitutions or compatibility issues.
\begin{itemize}
    \item This template uses \textbf{Open Sans} which has wider spacing and offers more clarity in a scientific setting than Theinhardt. Additionally, it's OFL licensed.
\end{itemize}
\end{itemize}
\end{frame}


\begin{frame}[fragile]{Chapter slides}
\begin{itemize}
\item Allow you to partition your presentation into multiple chapters
\item Also \texttt{frame}s, but with a few more options
\item Created with \verb|\begin{chapter}[<image>]{<color>}{<title>}|
  \begin{itemize}
  \item Image is optional, color and title are mandatory.
  \end{itemize}
\item Multiple background images have been provided in this template which may be used for chapter slides:
  \begin{itemize}
  \item Some campus photos that I have taken over the years and;
  \item some departmental logos.
  \item Let's look at a few examples!
  \end{itemize}
\end{itemize}
\end{frame}


\begin{chapter}[images/uic_seo.jpg]{uicblue}{Title goes here}
\textit{Subtitle goes here}
\end{chapter}

\themecolor{lightnologo}
% these backgrounds already have the UIC circular logo, so we temporarily change the theme to lightnologo
\begin{chapter}[images/dept_ece.png]{uicblue}{Title goes here}
\textit{Subtitle goes here}
\end{chapter}

\begin{chapter}[images/dept_cs_white.png]{uicblue}{Title goes here}
\textit{Subtitle goes here}
\end{chapter}

\begin{chapter}[images/dept_mscs.png]{expowhite}{Title goes here}
\textit{Subtitle goes here}
\end{chapter}
\themecolor{light}

\begin{frame}[fragile]{Other departments}
\begin{itemize}
\item You can use the generic \texttt{uic\_lockup.png} background, or
\item Download your logo from \hrefcol{https://marketing.uic.edu/marketing-toolbox/university-style-guide/university-marks-logos/}{here}.
    \begin{itemize}
    \item What you need is ``departmental lockup in SVG format and RGB/inverted RGB color".
    \end{itemize}
\item Overlay it on a static background (LaTeX cannot natively handle SVG images)
    \begin{itemize}
    \item Use the \texttt{overlay\_svg.py} Python script provided in the GitHub repository of this template (link on next slide)
    \end{itemize}
\end{itemize}
\end{frame}


\begin{frame}{Good luck with your presentation!}
\begin{itemize}
\item This template is hosted on GitHub
    \begin{itemize}
    \item  \texttt{https://github.com/usamamuneeb/uic-beamer-template}
    \end{itemize}
\item  I would appreciate contributions of all sorts (pull requests, identifying issues, etc).
\item  If you have any suggestions,
\hrefcol{mailto:umunee2@uic.edu}{send them to me!}
\end{itemize}
\end{frame}


\backmatter
\end{document}
